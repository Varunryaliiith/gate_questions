% \iffalse
\let\negmedspace\undefined
\let\negthickspace\undefined
\documentclass[beamer]{IEEEtran}
\usepackage{cite}
\usepackage{amsmath,amssymb,amsfonts,amsthm}
\usepackage{algorithmic}
\usepackage{graphicx}
\usepackage{textcomp}
\usepackage{xcolor}
\usepackage{txfonts}
\usepackage{listings}
\usepackage{enumitem}
\usepackage{mathtools}
\usepackage{gensymb}
\usepackage{comment}
\usepackage[breaklinks=true]{hyperref}
\usepackage{tkz-euclide} 
\usepackage{listings}
\usepackage{gvv}                                        
\def\inputGnumericTable{}                                 
\usepackage[latin1]{inputenc}                         
\usepackage{circuitikz}
\usepackage{color}                                            
\usepackage{array}                                            
\usepackage{longtable}                                       
\usepackage{calc}                                             
\usepackage{multirow}                                         
\usepackage{hhline}                                           
\usepackage{ifthen}                                           
\usepackage{lscape}
\usepackage[export]{adjustbox}

\newtheorem{theorem}{Theorem}[section]
\newtheorem{problem}{Problem}
\newtheorem{proposition}{Proposition}[section]
\newtheorem{lemma}{Lemma}[section]
\newtheorem{corollary}[theorem]{Corollary}
\newtheorem{example}{Example}[section]
\newtheorem{definition}[problem]{Definition}
\newcommand{\BEQA}{\begin{eqnarray}}
\newcommand{\EEQA}{\end{eqnarray}}
\newcommand{\define}{\stackrel{\triangle}{=}}
\theoremstyle{remark}
\newtheorem{rem}{Remark}
\begin{document}
\parindent 0px
\bibliographystyle{IEEEtran}

\title{GATE 2023 - EC 50}
\author{EE23BTECH11220 - R.V.S.S Varun$^{}$% <-this % stops a space
}
\maketitle
\newpage
\bigskip

\renewcommand{\thefigure}{\theenumi}
\renewcommand{\thetable}{\theenumi}
\section*{Question}

Let $x_1\brak{t}$ and $x_2\brak{t}$ be two band-limited signals having bandwidth B = $4\pi\times10^3$
rad/s each. In the figure below, the Nyquist sampling frequency, in
rad/s, required to sample y\brak{t}, is
  \\
\begin{figure}[h]
    \centering
    \includegraphics[width=0.5\linewidth]{figure_final.pdf}
    \label{fig:enter-label}
\end{figure}

    \brak{a} $20\pi\times10^3$\\
    \brak{b} $40\pi\times10^3$\\
    \brak{c} $8\pi\times10^3$\\
    \brak{d} $32\pi\times10^3$   \hfill(GATE EC 50)\\




\section*{Solution}


\begin{table}[h]
    \centering
    \begin{tabular}{|c|c|}
    \hline
        Symbol &Description \\
        \hline
        Y\brak{f}&y\brak{t} in frequency domain \\
        \hline
         $\omega_{m}$&Maximum frequency of Y\brak{f} \\
         \hline
         $\omega_{s}$&Nyquist sampling rate \\
         \hline
    \end{tabular}
    \caption{Table of parameters}
    \label{tab:my_label}
\end{table}
\text{$x_{1}\brak{t}$ and $x_{2}\brak{t}$ in frequency domain ,}
\begin{circuitikz}
    \draw[->] (-5,0) to (5,0);
    \draw[->] (0,-1) to (0,5);
    \draw (-3,0) to (0,3);
    \draw (0,3) to (3,0);
    \draw (-3,-0.5) node {$-4\pi\times10^3$};
    \draw (3,-0.5) node {$4\pi\times10^3$};
    \draw (0,5.5) node {$X_{1}\brak{f}$};

\end{circuitikz}

\begin{circuitikz}
    \draw[->] (-5,0) to (5,0);
    \draw[->] (0,-1) to (0,5);
    \draw (-3,0) to (0,3);
    \draw (0,3) to (3,0);
    \draw (-3,-0.5) node {$-4\pi\times10^3$};
    \draw (3,-0.5) node {$4\pi\times10^3$};
    \draw (0,5.5) node {$X_{2}\brak{f}$};
\end{circuitikz}
 From figure ,
 \begin{align}
     y\brak{t}=x_1\brak{t}cos\brak{4\pi\times10^3t}+x_2\brak{t}cos\brak{12\pi\times10^3t}
 \end{align}

 \begin{circuitikz}
    \draw[->] (-5,0) to (5,0);
    \draw[->] (0,-1) to (0,5);
    \draw (-4,0) to (-3,2);
    \draw (-3,2) to (-2,0);
    \draw (-2,0) to (-1,2);
    \draw (-1,2) to (0,0);
    \draw (0,0) to (1,2);
    \draw (1,2) to (2,0);
    \draw (2,0) to (3,2);
    \draw (3,2) to (4,0);
    \draw (-2,-0.5) node {$-8\pi\times10^3$};
    \draw (2,-0.5) node {$8\pi\times10^3$};
    \draw (4,-0.5) node {$16\pi\times10^3$};
    \draw (-4,-0.5) node {$-16\pi\times10^3$};
    \draw (0,5.5) node {$Y\brak{f}$};
\end{circuitikz}
\begin{center}
    y\brak{t} in frequency domain
\end{center}
\begin{align}
\omega_{m}=16\pi\times10^3 rad/sec.
\end{align}
\begin{align}
\omega_{s}=2\omega_{m}=32\pi\times10^3 rad/sec.
\end{align}
\end{document}
