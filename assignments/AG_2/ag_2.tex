% \iffalse
\let\negmedspace\undefined
\let\negthickspace\undefined
\documentclass[beamer]{IEEEtran}
\usepackage{cite}
\usepackage{amsmath,amssymb,amsfonts,amsthm}
\usepackage{algorithmic}
\usepackage{graphicx}
\usepackage{textcomp}
\usepackage{xcolor}
\usepackage{txfonts}
\usepackage{listings}
\usepackage{enumitem}
\usepackage{mathtools}
\usepackage{gensymb}
\usepackage{comment}
\usepackage[breaklinks=true]{hyperref}
\usepackage{tkz-euclide} 
\usepackage{listings}
\usepackage{gvv}                                        
\def\inputGnumericTable{}                                 
\usepackage[latin1]{inputenc}                                
\usepackage{color}                                            
\usepackage{array}                                            
\usepackage{longtable}                                       
\usepackage{calc}                                             
\usepackage{multirow}                                         
\usepackage{hhline}                                           
\usepackage{ifthen}                                           
\usepackage{lscape}
\usepackage[export]{adjustbox}

\newtheorem{theorem}{Theorem}[section]
\newtheorem{problem}{Problem}
\newtheorem{proposition}{Proposition}[section]
\newtheorem{lemma}{Lemma}[section]
\newtheorem{corollary}[theorem]{Corollary}
\newtheorem{example}{Example}[section]
\newtheorem{definition}[problem]{Definition}
\newcommand{\BEQA}{\begin{eqnarray}}
\newcommand{\EEQA}{\end{eqnarray}}
\newcommand{\define}{\stackrel{\triangle}{=}}
\theoremstyle{remark}
\newtheorem{rem}{Remark}
\begin{document}
\parindent 0px
\bibliographystyle{IEEEtran}

\title{GATE 2021 AG -Q.2}
\author{EE23BTECH11220 - R.V.S.S Varun$^{}$% <-this % stops a space
}
\maketitle
\newpage
\bigskip

\renewcommand{\thefigure}{\theenumi}
\renewcommand{\thetable}{\theenumi}
\section*{Question}
If x is an integer with  x$>$1, the solution of 
\begin{align*}
\lim_{x\to\infty}\left(\frac{1}{x^2}+\frac{2}{x^2}+\frac{3}{x^2}+\cdots+\frac{x-1}{x^2}+\frac{1}{x}\right)
\end{align*}
\begin{enumerate}[label=\alph*)]
\item Zero 
\item 0.5 
\item 1.0 
\item $\infty$  \hfill{\brak{GATE\ 2021\ AG\ Q.2}}
\end{enumerate}
\section*{Solution}

\begin{table}[h]
    \centering
\begin{tabular}{|c|c|c|}
    \hline
	Symbol &Description&Value \\
        \hline
	$f_1$&Frequency of cos$\brak{4\pi\times10^3}$&$2\times10^3$ \\
        \hline
	$f_2$&Frequency of cos$\brak{12\pi\times10^3}$&$6\times10^3$ \\
	\hline
	$f_m$&Maximum frequency of the output signal&- \\
	\hline
	 $\omega_{m}$&-&$2\pi f_m$ \\
         \hline
	 $\omega_{s}$&Nyquist sampling rate&$2\omega_m$ \\
         \hline
    \end{tabular}

 \caption{Table of parameters}
    \label{tab:AG.2.1}
\end{table}

From table ,
\begin{align}
	P\brak{z}=\frac{1}{x^2}\sum_{n=-\infty}^{n=\infty}\brak{n+1}u\brak{n}z^{-n} \label{AG.2.1}
\end{align}
\begin{align}
	n u\brak{n}\xleftrightarrow{\mathcal{Z}} \frac{z^{-1}}{\brak{1-z^{-1}}^2} ,   \abs{z} >1 \\
   u\brak{n}\xleftrightarrow{\mathcal{Z}} \frac{1}{\brak{1-z^-1}} ,   \abs{z} >1  
\end{align}
From \eqref{AG.2.1}
\begin{align}
  P\brak{z}=\frac{1}{x^2}\frac{1}{\brak{1-z^{-1}}^2} , \abs{z} >1\\
	q\brak{n}&=p\brak{n}\ast u\brak{n}\\
	\implies Q\brak{z}&=P\brak{z}U\brak{z}   \\
	 Q\brak{z}&=\brak{\frac{1}{x^2}\frac{1}{({1-z^{-1})}^{2}}}\brak{\frac{1}{1-z^{-1}}}  \\
	 &=\frac{1}{x^2}\frac{1}{({1-z^{-1})}^{3}} ,\quad \abs{z}>1
\end{align}
Using Contour Integration to find the inverse $Z$-transform,
\begin{align}
    q\brak{x-1}&=\frac{1}{2\pi j}\oint_{C}Q(z) \;z^{x-2} \;dz  \\
    &=\frac{1}{2\pi j}\oint_{C}\frac{z^{x-2}}{x^2\brak{1-z^{-1}}^{3}} \;dz 
\end{align}
We can observe that the pole is repeated $3$ times and thus $m=3$,
\begin{align}
    R&=\frac{1}{\brak {m-1}!}\lim\limits_{z\to a}\frac{d^{m-1}}{dz^{m-1}}\brak {{(z-a)}^{m}f\brak z} \\
    R&=\frac{1}{\brak {2}!}\lim\limits_{z\to 1}\frac{d^{2}}{dz^{2}}\brak {{(z-1)}^{3}\frac{z^{x+1}}{x^2\brak{z-1}^3}} \\
    R&=\frac{1}{\brak {2}!}\lim\limits_{z\to 1}\frac{d^{2}}{dz^{2}} \frac{z^{x+1}}{x^2} \\
    R&=\frac{1}{2}\frac{\brak{x+1}\brak{x}}{x^2} 
\end{align}
\begin{align}
    \lim_{x\to\infty}q\brak{x-1}=\frac{1}{2}
\end{align}
Hence , option \brak{B} is correct.
\end{document}

